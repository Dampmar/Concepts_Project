\documentclass{article}
\usepackage[a4paper, margin=1in]{geometry} % Adjust margin size as needed
\usepackage{graphicx} % Required for inserting images
\usepackage{listings}
\lstset{basicstyle=\ttfamily}
\usepackage{float}
\usepackage{courier}
\usepackage{tabularx}
\usepackage{tikz}
\usepackage{url}
\usepackage{amsmath}
\usepackage{float}
\usepackage{hyperref}
\usepackage{MnSymbol}
\usepackage{indentfirst}

\begin{document}
\title{Extended Project
\author{Daniel Marin and Jennifer Vicentes}
\date{November 16$^{th}$, 2024}}
\maketitle
\tableofcontents
\newpage

\section{Introduction}
The purpose of this project was to put in practice de concepts seen in class, by creating a fully fledged chord calculator based on the subject of parsing primarily and the concepts of programming as a whole.
\subsection{Grammar Rules}
\begin{figure}[H]
    \begin{centering}
    \begin{lstlisting}
            input:=  song EOF
             song:=  bar {bar} "|"
              bar:=  [meter] chords "|"
            meter:=  numerator "/" denominator
        numerator:=  "1" | "2" | "3" | ... | "15"
      denominator:=  "1" | "2" | "4" | "8" | "16"
           chords:=  "NC" | "%" | chord {chord}
            chord:=  root [description] [bass]
             root:=  note
             note:=  letter [acc]
           letter:=  "A" | "B" | "C" | ... | "G" |
              acc:=  "#" | "b"
      description:=  [qual] [qnum] [add] [sus] [omit]
      /* at least one, but not both qual & sus; see quiz 8 */
             qual:=  "-" | "+" | "o" | "5" | "1"
             qnum:=  "6" | ["^"] "7" | ["^"] ext
              ext:=  "9" | "11" | "13"
              add:=  alt | "(" alt ")"
              alt:=  [acc] "5" | [acc] ext
              sus:=  "sus2" | "sus4" | "sus24"
             omit:=  "no3" | "no5" | "no35"
             bass:=  "/" note
    \end{lstlisting}
    \caption{Chords grammar. The input consists of the chords in a song separated by bar lines ``$\vert$'' plus optional meter information for the bars. Figure 2 shows an example. (This grammar is a subset of the grammar used by Polynizer: \url{https://www.polynizer.com}.)}
    \label{fig:grammar}
    \end{centering}
\end{figure}

\subsection{First And Follow Sets}
\begin{figure}[H]
    \def\arraystretch{1.5}%
    \begin{tabularx}{\textwidth}{l|X}
        Nonterminal & First set \\
        \hline 
        \hline 
        \lstinline|input| & \{1, 2, 3, \ldots, 15, A, B, C, D, E, F, G, \lstinline|NC|, \%\} \\ 
        \hline 
        \lstinline|song| & \{1, 2, 3, \ldots, 15, A, B, C, D, E, F, G, \lstinline|NC|, \%\} \\ 
        \hline
        \lstinline|bar| & \{1, 2, 3, \ldots, 15, A, B, C, D, E, F, G, \lstinline|NC|, \%\} \\ 
        \hline 
        \lstinline|meter| & \{1, 2, 3, \ldots, 15\} \\ 
        \hline 
        \lstinline|numerator| & \{1, 2, 3, \ldots, 15\} \\ 
        \hline 
        \lstinline|denominator| & \{1, 2, 4, 8, 16\} \\ 
        \hline 
        \lstinline|chords| & \{A, B, C, D, E, F, G, \lstinline|NC|, \%\} \\
        \hline
        \lstinline|chord| & \{A, B, C, D, E, F, G\} \\
        \hline 
        \lstinline|root| & \{A, B, C, D, E, F, G\} \\
        \hline
        \lstinline|note| & \{A, B, C, D, E, F, G\} \\
        \hline
        \lstinline|letter| & \{A, B, C, D, E, F, G\} \\
        \hline 
        \lstinline|acc| & \{\#, b\}\\
        \hline 
        \lstinline|description| &\\
        \hline 
        \lstinline|qual| & \{-, +, o, 5, 1\}\\
        \hline 
        \lstinline|qnum| & \{6, 7, 9, 11, 13, \lstinline|^|\}\\
        \hline 
        \lstinline|ext| & \{9, 11, 13\} \\
        \hline 
        \lstinline|add| & \\
        \hline 
        \lstinline|alt| & \\
        \hline 
        \lstinline|sus| & \{\texttt{sus2}, \texttt{sus4}, \texttt{sus24}\} \\
        \hline 
        \lstinline|omit| & \{\texttt{no2}, \texttt{no5}, \texttt{no35}\} \\
        \hline 
        \lstinline|bass| & \{/\} \\
        \hline
    \end{tabularx}
\end{figure}

\end{document}